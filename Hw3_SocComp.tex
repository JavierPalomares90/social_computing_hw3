%
% This is the LaTeX template file for lecture notes for EE 382C/EE 361C.
%
% To familiarize yourself with this template, the body contains
% some examples of its use.  Look them over.  Then you can
% run LaTeX on this file.  After you have LaTeXed this file then
% you can look over the result either by printing it out with
% dvips or using xdvi.
%
% This template is based on the template for Prof. Sinclair's CS 270.

\documentclass[twoside]{article}
\usepackage{graphics}
\usepackage{tikz}
\usepackage{pgfplots}
\pgfplotsset{compat=1.15}
\usetikzlibrary{intersections}
\usetikzlibrary{patterns}
\setlength{\oddsidemargin}{0.25 in}
\setlength{\evensidemargin}{-0.25 in}
\setlength{\topmargin}{-0.6 in}
\setlength{\textwidth}{6.5 in}
\setlength{\textheight}{8.5 in}
\setlength{\headsep}{0.75 in}
\setlength{\parindent}{0 in}
\setlength{\parskip}{0.1 in}

%
% The following commands set up the lecnum (lecture number)
% counter and make various numbering schemes work relative
% to the lecture number.
%
\newcounter{lecnum}
\renewcommand{\thepage}{\thelecnum-\arabic{page}}
\renewcommand{\thesection}{\thelecnum.\arabic{section}}
\renewcommand{\theequation}{\thelecnum.\arabic{equation}}
\renewcommand{\thefigure}{\thelecnum.\arabic{figure}}
\renewcommand{\thetable}{\thelecnum.\arabic{table}}

%
% The following macro is used to generate the header.
%
\newcommand{\drawle}{-- (rel axis cs:1,1) -- (rel axis cs:1,0) -- (rel axis cs:0,0) \closedcycle}

\newcommand{\lecture}[4]{
   \pagestyle{myheadings}
   \thispagestyle{plain}
   \newpage
   \setcounter{lecnum}{#1}
   \setcounter{page}{1}
   \noindent
   \begin{center}
   \framebox{
      \vbox{\vspace{2mm}
    \hbox to 6.28in { {\bf EE 382V: Social Computing
                        \hfill Fall 2018} }
       \vspace{4mm}
       \hbox to 6.28in { {\Large \hfill Homework 2: #2  \hfill} }
       \vspace{2mm}
       \hbox to 6.28in { {\it Partner1: #3 \hfill Partner2: #4} }
      \vspace{2mm}}
   }
   \end{center}
   \markboth{EE382V:Social Computing HW3: #2}{EE382V:Social Computing HW3: #2}
   %{\bf Disclaimer}: {\it These notes have not been subjected to the
   %usual scrutiny reserved for formal publications.  They may be distributed
   %outside this class only with the permission of the Instructor.}
   \vspace*{4mm}
}

%
% Convention for citations is authors' initials followed by the year.
% For example, to cite a paper by Leighton and Maggs you would type
% \cite{LM89}, and to cite a paper by Strassen you would type \cite{S69}.
% (To avoid bibliography problems, for now we redefine the \cite command.)
% Also commands that create a suitable format for the reference list.
\renewcommand{\cite}[1]{[#1]}
\def\beginrefs{\begin{list}%
        {[\arabic{equation}]}{\usecounter{equation}
         \setlength{\leftmargin}{2.0truecm}\setlength{\labelsep}{0.4truecm}%
         \setlength{\labelwidth}{1.6truecm}}}
\def\endrefs{\end{list}}
\def\bibentry#1{\item[\hbox{[#1]}]}

%Use this command for a figure; it puts a figure in wherever you want it.
%usage: \fig{NUMBER}{SPACE-IN-INCHES}{CAPTION}
\newcommand{\fig}[3]{
			\vspace{#2}
			\begin{center}
			Figure \thelecnum.#1:~#3
			\end{center}
	}
% Use these for theorems, lemmas, proofs, etc.
\newtheorem{theorem}{Theorem}[lecnum]
\newtheorem{lemma}[theorem]{Lemma}
\newtheorem{proposition}[theorem]{Proposition}
\newtheorem{claim}[theorem]{Claim}
\newtheorem{corollary}[theorem]{Corollary}
\newtheorem{definition}[theorem]{Definition}
\newenvironment{proof}{{\bf Proof:}}{\hfill\rule{2mm}{2mm}}

% **** IF YOU WANT TO DEFINE ADDITIONAL MACROS FOR YOURSELF, PUT THEM HERE:

\begin{document}
%FILL IN THE RIGHT INFO.
%\lecture{**LECTURE-NUMBER**}{**DATE**}{**LECTURER**}{**SCRIBE**}
\lecture{1}{November 9}{Javier Palomares}{Porter Perry}
%\footnotetext{These notes are partially based on those of Nigel Mansell.}

% **** YOUR NOTES GO HERE:

% Some general latex examples and examples making use of the
% macros follow.  
%**** IN GENERAL, BE BRIEF. LONG SCRIBE NOTES, NO MATTER HOW WELL WRITTEN,
%**** ARE NEVER READ BY ANYBODY.
\section{Question 1}
Say whether the claim is true or false with a brief justification.

(a) If player A in a two-person game has a dominant strategy $S_{A}$, then there is a pure strategy Nash equilibrium in which player A plays $S_{A}$ and player B plays a best response to $S_{A}$.\\
(b) In a Nash equilibrium of a two-player game each player is playing an optimal strategy, so the two players strategies are social-welfare maximizing.

\subsection{Answer 1a}

True. If player A is playing a dominant strategy $S_A$ then player A will not increase his payout by playing a 
different strategy. Then if player B is playing the best response to $S_A$, player's B payout will not increase 
by playing a different strategy. Since neither player's payout increases by playing a different strategy, their strategies for a Nash equilibrium.player A's and player B's payout does not increase by playing a different strategy.

\subsection{Answer 1b}

False. Consider the Prisoner's Dilemna game 

\hspace*{40mm}\texttt{Player B }\\
\hspace*{40mm}\texttt{C} \hspace*{15mm}\texttt{NC}\\
\hspace*{24mm}\texttt{C} \hspace{12mm}\texttt{-1,-1} \hspace{12mm}\texttt{-10,0}\\
\texttt{Player A} \hspace*{8mm}\texttt{NC} \hspace{12mm}\texttt{0,-10} \hspace{12mm}\texttt{-4,-4}\\

There is a Nash Equilibrium at (C,C) = (-4,4) but the socially optimal solution is (N,C) =(-1,-1).

\section{Question 2}
Find all pure strategy Nash equilibria in the game below. In the payoff matrix below the rows correspond to player A's strategies and columns correspond to player B's strategies. The first entry in each box is player A's payoff and the second entry is player B's payoff.

\hspace*{40mm}\texttt{Player B }\\
\hspace*{40mm}\texttt{L} \hspace*{15mm}\texttt{R}\\
\hspace*{24mm}\texttt{U} \hspace{12mm}\texttt{1,2} \hspace{12mm}\texttt{3,2}\\
\texttt{Player A} \hspace*{8mm}\texttt{D} \hspace{12mm}\texttt{2,4} \hspace{12mm}\texttt{0,2}\\

\subsection{Answer 2}

Pure strategy Nash Equilibrias: \\
\texttt{DL = (2,4)}\\
\texttt{UR = (3,2)}

\section{Question 3}
In this question we will consider several two-player games.\\

(a) Find all pure (non-randomized) strategy Nash equilibria for the game described by the payoff matrix below.\\

\hspace*{40mm}\texttt{Player B }\\
\hspace*{40mm}\texttt{L} \hspace*{15mm}\texttt{R}\\
\hspace*{24mm}\texttt{U} \hspace{11mm}\texttt{2,15} \hspace{11mm}\texttt{4,20}\\
\texttt{Player A} \hspace*{8mm}\texttt{D} \hspace{11mm}\texttt{6,6} \hspace{12mm}\texttt{10,8}\\

(b) Find all pure (non-randomized) strategy Nash equilibria for the game described by the payoff matrix below.\\

\hspace*{40mm}\texttt{Player B }\\
\hspace*{40mm}\texttt{L} \hspace*{15mm}\texttt{R}\\
\hspace*{24mm}\texttt{U} \hspace{12mm}\texttt{3,5} \hspace{12mm}\texttt{4,3}\\
\texttt{Player A} \hspace*{8mm}\texttt{D} \hspace{12mm}\texttt{2,1} \hspace{12mm}\texttt{1,6}\\

(c) Find all Nash equilibria for the game described by the payoff matrix below.\\

\hspace*{40mm}\texttt{Player B }\\
\hspace*{40mm}\texttt{L} \hspace*{15mm}\texttt{R}\\
\hspace*{24mm}\texttt{U} \hspace{12mm}\texttt{1,1} \hspace{12mm}\texttt{4,2}\\
\texttt{Player A} \hspace*{8mm}\texttt{D} \hspace{12mm}\texttt{3,3} \hspace{12mm}\texttt{2,2}\\

\subsection{Answer 3a}

Pure strategy Nash Equilibria: \\
\texttt{DR = (10,8)}

\subsection{Answer 3b}

Pure strategy Nash Equilibria: \\
\texttt{UL = (3,5)}

\subsection{Answer 3c}

Pure strategy Nash Equilibria: \\
\texttt{UR = (4,2)}
\texttt{DL = (3,3)}

Let $p$ represent the probability that Player A chooses strategy $S_{U}$, and $(1-p)$ represent the probability that Player A chooses strategy $S_{D}$. Likewise, let $q$ represent the probability that Player B chooses strategy $T_{L}$, and $(1-q)$ represent the probability that Player B chooses strategy $T_{R}$.

Player A expected payoff for Strategy $S_{U}$: $1*q + 4*(1-q) = 4 - 3q$ \\
Player A expected payoff for Strategy $S_{D}$: $3*q + 2*(1-q) = 2 + q$ \\
Indifference: 
$4 - 3q = 2 + q$ \\
$2 = 4q$ \\
$q = \frac{1}{2}$ \\

Player B expected payoff for Strategy $S_{L}$: $1*p + 3*(1-p) = 3 - 2p$ \\
Player B expected payoff for Strategy $S_{R}$: $2*p + 2*(1-p) = 2$ \\
Indifference: 
$3 - 2p = 2$ \\
$1 = 2p$ \\
$p = \frac{1}{2}$ \\

Mixed strategy Nash equilibria probabilistic vector: \\
$((\frac{1}{2}, \frac{1}{2}), (\frac{1}{2}, \frac{1}{2}))$

\end{document}
